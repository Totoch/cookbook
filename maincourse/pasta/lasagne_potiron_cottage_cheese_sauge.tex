\begin{recipe}[
	source = Spar,
	portion = 6,
	bakingtime = 1 heure 30 minutes,
	bakingtemperature = {\protect\bakingtemperature{
		fanoven=\unit[140]{ºC}
		}
	},
]{Lasagne de potiron, cottage cheese et sauge}
\ingredients[15]{
	\unit[1.2]{kg} & Butternut \\
	\unit[1]{dl} & Crème Boni sélection \\
	\unit[600]{g} & Cottage cheese Danone \\
	3 & Oeufs \\
	\unit[150]{g} & Parmesan râpé \\
	2 & Paquets de jambon italien \\
	 & Feuilles de lasagnes fraîches \\
	\unit[20]{g} & Noisettes grillées Spar Tasty \\
	1 & Plante de sauge \\
	& Beurre \\
	& Huile d'olive \\
	& Pili-pili \\
	& Noix de muscade \\
}

	\index{Source!Spar}
	\index{Plat!Pâtes!Spar}
	\index{Plat!Pâtes!Lasagne}
	\index{Ingrédients!Fromages!Cottage cheese}
	\index{Ingrédients!Légumes!Butternut}
	\index{Ingrédients!Légumes!Potiron}


\preparation {
	\step Préchauffer le four à \unit[200]{°C}.
	\step Couper le potiron en deux dans le sens de la longueur et ôter les pépins. Badigeonner la chair d'huile d'olive. Assaisonner de sel marin, poivre noir du moulin et pili-pili. Placer le potiron 50 minutes dans le four jusqu'à ce que la chair soit tendre.
	\step Évider le potiron à la cuillère. Couper la chair en petits morceaux et metter-les dans une casserolle avec une noix de beurre et 5 feuilles de sauge finement ciselées. Faites revenir un moment. Ajouter la crème et réduiser le tout en purée. Rectifier l'assaisonnement avec du sel et du pili-pili.
	\step Batter les oeufs avec le cottage cheese. Ajouter-y le parmesan râpé. Relever de poivre et de noix de muscade.
	\step Verser une fine couche de purée de potiron dans le fond d'un plat allant au four, que vous aver préalablement badigeonné d'huile. Couvrer d'une couche de feuilles de lasagne et d'une couche de cottage cheese. Recouvrer de fines tranches de jambon italien et d'une nouvelle couche de purée de potiron, de lasagne, et ainsi de suite.
	\step Répéter jusqu'à épuisement des ingrédients et terminer par une couche de cottage cheese.
	\step Placer le plat pendant 30 minutes dans le four à \unit[180]{°C}, jusqu'à ce que la surface soit bien dorée.
	\step Entre-temps, hacher grossièrement les noisettes. Faites rissoler quelques feuilles de sauge dans du beurre, avec le reste de jambon italien, jusqu'à ce que tout soit croquant.
	\step Décorer la lasagne avec la sauge, le jambon et les noisettes. Server chaud.
}
\end{recipe}
