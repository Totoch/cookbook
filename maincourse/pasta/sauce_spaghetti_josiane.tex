\begin{recipe}[
 bakingtime = 1 heure,
 source = Josiane Cerfontaine
]{Sauce spaghetti Josiane}
\label{SauceSpaghettiJosiane}
	\ingredients[22]{
	\unit[1]{kg} & Hachis de porc/boeuf ou porc/veau \\
	3 & Gros oignons \\
	2 & Éclats d'ail \\
	\unit[800]{g} & Tomates concassées \\
	\unit[140]{g} & Concentré de tomates \\
	3 & Cubes de bouillon de légumes \\
	\unit[7]{c à c} & Maïzena \\
	& Huile d'olive \\
	Épices: & \\
	1 & Sucre blanc \\
	& Sel \\
	& Poivre \\
	& Pili-pili \\
	1-2 & Feuilles de laurier \\
	2 & Clous de girofle \\
	& Romarin \\
	& Marjolaine \\
	& Basilic \\
	& Origan \\
	& Thym \\
	& Herbes de provence \\
	2 & Feuilles de sauge \\
}

	\index{Source!Josiane Cerfontaine}
	\index{Ingrédients!Légumes!Boîte de tomates}
	\index{Plat!Pâtes!Sauce}
	\index{Plat!Pâtes!Josiane}
	\index{Ingrédients!Viandes!Hachis de porc}
	\index{Ingrédients!Viandes!Hachis de boeuf}
	\index{Ingrédients!Viandes!Hachis de veau}


\preparation {
	\step Mettre fondre un peu de beurre avec un peu d'huile d'olive dans une grande poêle.
	\step Faire revenir la viande avec la sauge jusqu'à ce qu'elle soit bien dorée.
	\step Dans une grande casserole, faire chauffer de l'huile d'olive et y faire revenir les oignons et l'ail.
	\step Ajouter tous les autres ingrédients et ensuite la viande avec son jus de cuisson.
	\step Amener à ébulition et laisser mijoter à feu doux pendant ± 1 heure.
	\step Mixer légèrement puis lier un peu avec de la maïzena diluée dans un peu d'eau froide.
	\step Refaire bouillir un peu.
}
\end{recipe}
