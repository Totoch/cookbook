\begin{recipe}[
	source = Delhaize,
	portion = 4,
	bakingtime = 30-35 minutes,
	bakingtemperature = {\protect\bakingtemperature{
		fanoven=\unit[160]{ºC}
		}
	},
	preparationtime = 1h
]{Cannellonis au fromage de chêvre, aux herbes vertes et au coulis de potiron}
\ingredients[14]{
	\unit[250]{g} & Fromage de chêvre frais \\
	\unit[250]{g} & Ricotta \\
	\unit[1]{kg} & Potiron \\
	\unit[10]{g} & Persil plat \\
	\unit[10]{g} & Menthe fraîche \\
	\unit[10]{g} & Thym frais \\
	\unit[30]{g} & Sauge fraîche \\
	1 & Oeuf \\
	\unit[200]{g} & Parmesan râpé \\
	\unit[2]{dl} & Crème \\
	\unit[330]{g} & Cubes de tomates \\
	20 & Cannellonis \\
	& Poivre \\
	& Sel \\
}

	\index{Source!Delhaize}
	\index{Plat!Pâtes!Delhaize}
	\index{Ingrédients!Fromages!Chêvre}
	\index{Ingrédients!Fromages!Ricotta}
	\index{Ingrédients!Fromages!Parmesan}
	\index{Ingrédients!Légumes!Potiron}


\preparation {
	\step Réserver la moitié de la sauge pour la finition.
	\step Hacher finement le restant des hebes vertes.
	\step Mélanger la ricotta et le fromage de chèvre avec 2/3 du parmesan, l'oeuf et les épices hachées.
	\step Poivrer et saler puis mélanger à nouveau.
	\step Couper le potiron en deux. Enlever les graines et couper la chair en morceau.
	\step Faites cuire la chair dans de l'eau bouillante salée pendant ± 15 minutes.
	\step Égoutter et mixer le potiron avec les cubes de tomates. Ajouter la crème, poivrer et saler.
	\step Préchauffer le four à \unit[160]{ºC}.
	\step Verser la moitié de la sauce tomates-potiron dans un plat à four. 
	\step Metter le mélange de fromage dans une poche à douille et remplisser-en les cannellonnis.
	\step Napper du restant de sauce et saupoudrer du reste de parmesan.
	\step Couvrer soigneusement d'une feuille de papier aluminium et faites cuire au four pendant 30/35 minutes. Retirer la feuille de papier aluminium 10 minutes avant la fin de la cuisson.
	\step Gratiner les cannellonis pendant quelques minutes.
	\step Hacher grossièrement le reste de la sauge et parsemer-en les cannellonis.
}
\end{recipe}
